\chapter{Projeto de estudo de Caso}

Esse capítulo irá tratar da estratégia de pesquisa adotada durante o trabalho, buscando estar de acordo com ...


\section{Definição sobre estudo de caso}

O estudo de caso é uma estratégia de pesquisa utilizada para investigar um tópico de manira empírica através de um conjunto de procedimentos pré-especificados \cite{yin2001estudo}. Buscando diferenciar o estudo de caso de outras estratégias de pesquisa, \citeonline{yin2001estudo} esclarece que um estudo de caso deve focalizar acontecimentos contemporâneos, não havendo assim exigência quanto ao controle sobre os eventos comportamentais. Dessa forma, o estudo de caso difere de um experimento pelo motivo que neste há controle e manipulação sobre os eventos, diferentemente do estudo de caso, que não os manipula. Em suma, \citeonline{schramm_notes_1971} define que a essência de qualquer estudo de caso reside em esclarecer uma decisão ou um conjunto de decisões, considerando o motivo pelo qual elas foram tomadas e qual os resultados das suas implementações \cite{schramm_notes_1971}. 

\section{Modelagem do estudo de caso}

Buscando maior entendimento a respeito do estudo de caso proposto por esse trabalho, foram criadas algumas perguntas que são fundamentais para o seu entendimento:

\begin{easylist}[itemize]	
	
	& Qual o problema a ser tratado?
	& Qual a questão de pesquisa relacionada a esse problema?
	& Quais são os objetivos a serem alcançados nessa pesquisa?	
	& Como foi a seleção do estudo de caso?
	& Qual fonte dos dados coletados nessa pesquisa?
	& Qual o método de coleta de dados?
	
	\end{easylist}	
	
As perguntas acima serão respondidas nas próximas seções, de modo que o estudo de caso possa ser compreendido como um projeto de pesquisa e então ser executado

\section{Problema}

O PROBLEMA


\section{Questão de Pesquisa}

Segundo \citeonline{caldiera_goal_1994}, a questão de pesquisa deve ser capaz de caracterizar o objeto que está sendo medido, seja ele produto, processo ou recurso. Sob essa lógica, a seguinte questão de pesquisa foi criada após análise do problema:

( ESCREVER QUESTÃO DE PESQUISA)

Para atender a questão de pesquisa foi utilizado o mecanismo goal-question-metrics (GQM), usado para definir e interpretar um software operacional e mensurável. O GQM combina em si muitas das técnicas de medição e as generaliza para incorporar processos, produtos ou recursos, o que torna seu uso adaptável a ambientes diferentes \cite{caldiera_goal_1994}. 


\subsection{Objetivos, questões e métricas}

\textbf{Objetivo 01:} (ESCREVER)

\textbf{Questão específica 01:} (ESCREVER)

\textbf{Fonte:} (ESCREVER)

\textbf{Métrica:} (ESCREVER)

\textbf{Questão específica 02:} (ESCREVER)

\textbf{Fonte:} (ESCREVER)

\textbf{Métrica:} (ESCREVER)

\textbf{Objetivo 02:} (ESCREVER)

\textbf{Questão específica 03:} (ESCREVER)

\textbf{Fonte:} (ESCREVER)

\textbf{Métrica:} (ESCREVER)


\textbf{Questão específica 04:} (ESCREVER)

\textbf{Fonte:} (ESCREVER)

\textbf{Métrica:} (ESCREVER)

\section{\textit{Background}}

\section{Seleção}

Os dados foram coletados no TCU porque...

\section{Fonte dos dados coletados}

O dados foram coletados via análise do código fonte e questionários que...

\section{Ameaças a validade do estudo de caso}

\section{Processo de análise dos dados}

\section{Conclusão do capítulo}

\label{estudo de caso}

