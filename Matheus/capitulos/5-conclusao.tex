\chapter{Conclusão}

A primeira etapa desse trabalho apresentou uma revisão bibliográfica a respeito de qualidade interna de software, sobre \textit{Data Warehousing} e acerca da definição de estudo de caso. Os conceitos levantados na revisão bibliográfica foram utilizados durante a apresentação da solução, que utiliza \textit{Data Warehousing} para monitoramento de métricas de código fonte. Após a revisão bibliográfica ficou clara a importância de monitorar código fonte e também quais as vantagens no uso de \textit{Data Warehousing} para armazenamento de dados em relação a sistemas OLTP. 

Após a apresentação sobre a revisão bibliográfica foi mostrado o projeto de pesquisa a ser realizado. Conceitos fundamentais a uma pesquisa como qual o problema a ser resolvido e a questão que o caracteriza foram identificados e apresentados. Em seguida, utilizou-se da  técnica GQM para, através de objetivos, questões específicas e métricas, responder a questão de pesquisa. As etapas pelas quais o estudo de caso atravessará foram descritas e modeladas de forma sequencial, de tal forma a obedecer a revisão bibliográfica sobre estudo de caso.

Enquanto esta parte do trabalho teve como foco o planejamento do estudo de caso, a próxima etapa será responsável pela coleta e análise dos dados a serem obtidos, analisando a eficácia e eficiência do uso de \textit{Data Warehousing} no monitoramento de métricas de código fonte em uma autarquia da administração pública federal, com o objetivo de responder a questão de pesquisa definida.

Na segunda parte desse trabalho será realizada também uma análise estatística da taxa de oportunidade de melhoria de código em projetos de software livre já utilizados no trabalho de \cite{Meirelles2013}. O objetivo dessa análise é obter um intervalo qualitativo para essa taxa de oportunidade, possibilitando assim que a eficácia da solução proposta possa ser aferida com mais facilidade e exatidão no estudo de caso. Além disso, serão criados mais cenários de limpeza relacionando-os a um conjunto de métricas de código.