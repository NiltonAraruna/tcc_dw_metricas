\chapter{Projeto de estudo de Caso}

Este capítulo irá apresentar a estratégia de pesquisa adotada durante o trabalho, buscando elaborar um protocolo para o estudo de caso que será realizado no TCU. Elementos de pesquisa como o problema a ser resolvido e quais são os objetivos a serem alcançados serão identificados e explicados. Também será apresentado o método para coleta dos dados e como eles serão analisados.


\section{Definição sobre estudo de caso}

O estudo de caso é uma estratégia de pesquisa utilizada para investigar um tópico de manira empírica através de um conjunto de procedimentos pré-especificados \cite{yin2001estudo}. Buscando diferenciar o estudo de caso de outras estratégias de pesquisa, \citeonline{yin2001estudo} esclarece que um estudo de caso deve focalizar acontecimentos contemporâneos, não havendo assim exigência quanto ao controle sobre os eventos comportamentais. Dessa forma, o estudo de caso difere de um experimento pelo motivo que neste há controle e manipulação sobre os eventos, diferentemente do estudo de caso, que não os manipula. Em suma, a essência de qualquer estudo de caso reside em esclarecer uma decisão ou um conjunto de decisões, considerando o motivo pelo qual elas foram tomadas e qual os resultados das suas implementações \cite{schramm_notes_1971}. 

Buscando maior entendimento a respeito do estudo de caso proposto por esse trabalho, foram criadas algumas perguntas que são fundamentais para o seu entendimento:

\begin{easylist}[itemize]	
	
	& Qual o problema a ser tratado?
	& Qual a questão de pesquisa relacionada a esse problema?
	& Quais são os objetivos a serem alcançados nessa pesquisa?	
	& Como foi a seleção do estudo de caso?
	& Qual fonte dos dados coletados nessa pesquisa e qual o método de coleta?
	
	\end{easylist}	
	
As perguntas acima serão respondidas nas próximas seções, de modo que o estudo de caso possa ser compreendido como um projeto de pesquisa e então ser executado. Para tanto a seguinte estrutura será adotada no projeto de pesquisa:

\begin{figure}[h!]
\centering
\includegraphics[keepaspectratio=false,scale=0.40]{figuras/figuras_matheus/projeto_de_pesquisa.eps}
\caption{Estrutura do estudo de caso}
\label{fig:pesquisa}
\end{figure}
\FloatBarrier

\subsection{Problema}

O PROBLEMA 


\subsection{Questão de Pesquisa}

Segundo \citeonline{caldiera_goal_1994}, a questão de pesquisa deve ser capaz de caracterizar o objeto que está sendo medido, seja ele produto, processo ou recurso. Sob essa lógica, a seguinte questão de pesquisa foi criada após análise do problema:

( ESCREVER QUESTÃO DE PESQUISA)

Para atender a questão de pesquisa foi utilizado o mecanismo goal-question-metrics (GQM), usado para definir e interpretar um software operacional e mensurável. O GQM combina em si muitas das técnicas de medição e as generaliza para incorporar processos, produtos ou recursos, o que torna seu uso adaptável a ambientes diferentes \cite{caldiera_goal_1994}. 


\textbf{Objetivo 01:} Avaliar a eficácia e eficiência do uso de \textit{Data Warehouse} para monitoramento de métricas de código fonte.

% Questão 01

\textbf{QE01:} Quantas tomadas de decisão foram realizadas pela equipe baseando-se no uso da solução desenvolvida em um <período de tempo>?

\textbf{Fonte:} Registro de observação em campo.

\textbf{Métrica:} Número de decisões tomadas/tempo.

% Questão 02

\textbf{QE02: } Quantas tomadas de decisão ao todo foram realizadas pela equipe baseando-se no uso da solução desenvolvida?

\textbf{Fonte:} Registro de observação em campo.

\textbf{Métrica:} Número de decisões tomadas.

% Questão 03

\textbf{QE03: } Qual a avaliação da equipe de qualidade quanto a detecção de cenários de limpeza de código?

\textbf{Fonte:} Questionário com equipe de qualidade.

\textbf{Métrica:} muito bom, bom, regular, ruim, muito ruim.

% Questão 04

\textbf{QE04: } Com que frequência a equipe de qualidade encontra falhas relacionados à utilização da ferramenta em um determinado intervalo de tempo?

\textbf{Fonte:} Registro de observação em campo.

\textbf{Métrica:} Quantidade de falha / tempo (release, sprint).

\textbf{Interpretação da métrica:} Quanto mais próximo de zero melhor 0=<X

% Questão 05

\textbf{QE05: } Qual a quantidade total de falhas encontradas pela equipe de qualidade relacionadas à utilização da ferramenta?

\textbf{Fonte:} Registro de observação em campo (áudio/vídeo).

\textbf{Métrica:} Quantidade de falhas.

% Questão 06

\textbf{QE06: } Qual a proporção do uso da ferramenta para tomada de decisões?

\textbf{Fonte:} Questionário com equipe de qualidade.

\textbf{Métrica:} Número de decisões tomadas / número de vezes que a solução foi usada.


% Questão 07

\textbf{QE07: } Qual a quantidade de cenários que foram corrigidos após utilização da solução?

\textbf{Fonte:} Código fonte.

\textbf{Métrica:} Números de cenários corrigidos / número de cenários encontrados.

% Questão 08

\textbf{QE08: } Qual o nível de satisfação do uso da solução em comparação à solução anterior? 

\textbf{Fonte:} Equipe de qualidade.

\textbf{Métrica:} Muito satisfeito, Satisfeito, Neutro, Insatisfeito, Muito Insatisfeito.

% Questão 09

\textbf{QE09: } Qual a taxa de oportunidade de melhoria de código da solução em uma intervalo de tempo (sprint, release)? 

\textbf{Fonte:} Código fonte.

\textbf{Métrica:} Taxa de oportunidade de melhoria de código: FÓRMULA, onde Ce é o total de cenários de limpezas identificados e C1 é o total de classes em um intervalo de tempo (sprint, release).

\textbf{Interpretação da métrica: } 

\begin{easylist}[itemize]	
	& Número de classes crescente e constante, Número de oportunidade de melhoria estável: Projeto cresceu mais dos 	que o cenários, cenários podem ou não estar sendo tratados.
	& Número de classes crescente e constante, Número de oportunidade de melhoria crescendo: Projeto cresce junto 		com cenários que não são tratados com eficiência ou não são tratados
	& Número de classes crescente ou não, mas constante, número de oportunidade de melhoria diminuindo: Projeto 		cresce e cenários são tratados com eficiência.	
	\end{easylist}	

\section{\textit{Background}}

No que se refere ao contexto desse trabalho, o principal \textit{background} é o trabalho desenvolvido por \citeonline{rego_monitoramento_2014}, no qual a solução para monitoramento de métricas de código fonte utilizando \textit{Data Warehouse} foi desenvolvida. \citeonline{noveloo_uma_2006} utilizu \textit{Data Warehouse} para monitoramento de métricas no processo de desenvolvimento de software, porém não essas métricas não eram de código fonte. Não foram encontrados outros trabalhos que usassem \textit{Data Warehouse} para aferir qualidade do que está sendo desenvolvido relacionando métricas de código fonte a cenários de limpeza, tal qual foi feito por \citeonline{rego_monitoramento_2014}.

\section{Seleção}

Os dados foram coletados no TCU porque ?????????????????????????????????

\section{Fonte dos dados coletados e método de coleta}

Nesse estudo de caso, os dados foram coletados através de registros de observações em campo, questionários e resultados gerados pela própria solução após análise direta do código fonte.

Os registros oriundos de observações em campo são coletados durante encontros realizados no próprio (TCU?) com a equipe responsável pela tomada de decisões de qualidade de código fonte. Nesses encontros, a solução proposta é utilizada e qualquer atitude relacionada ao seu uso pela equipe é registrado.

A adoção de questionários foi utilizada tanto para dados qualitativos quanto para dados quantitativos. Um exemplo disso é a questão específica 06, em que se busca saber a proporção de decisões tomadas influenciadas pela solução via questionário com a empresa. O uso de questionário na questão 06 é quantitativo, enquanto na questão 08 é feito um questionário para saber o nível de satisfação da empresa quanto ao uso da solução, sendo esse um tipo de dado qualitativo.

Dados resultantes da própria ferramenta, como gráficos automaticamente gerados, serão coletados a medida que a ferramenta for utilizada ao longo do tempo.

\section{Ameaças a validade do estudo de caso}

\citeonline{yin2001estudo} descreve como principais ameaças relacionadas à validade do estudo de caso as ameaças relacionadas à validade do constructo, à validade interna, à validade externa e à confiabilidade. As quatro ameaças definidas por ele, bem como a forma usada nesse trabalho para preveni-las, são descritas da seguinte maneira: 

\begin{easylist}[itemize]	

& \textbf{Validade do Constructo: } A validade de construção está presente na fase de coleta de dados, quando deve ser evidenciado as múltiplas fontes de evidência e a coleta de um conjunto de métricas para que se possa saber exatamente o que medir e quais dados são relevantes para o estudo, de forma a responder as questões de pesquisa \cite{yin2001estudo}. Buscou-se garantir a validade de construção ao definir objetivos com evidências diferentes. Estas, por sua vez, estão diretamente relacionadas com os objetivos do estudo de caso e os objetivos do trabalho. 

& \textbf{Validade interna: } Para \citeonline{yin2001estudo} o uso de várias fontes de dados e métodos de coleta permite a triangulação, uma técnica para confirmar se os resultados de diversas fontes e de diversos métodos convergem. Dessa forma é possível aumentar a validade interna do estudo e aumentar a força das conclusões.
A triangulação de dados se deu pelo resultado da solução de \textit{Data Warehouse} que utiliza o código-fonte (explicada no capítulo \ref{chap:arquitetura} ) , pela análise de questionários e pelos dados coletados através de entrevistas.

& \textbf{Validade externa: } Por este ser um caso único, a generalização do estudo de caso se dá de maneira pobre \cite{yin2001estudo}. Assim é necessária a utilização do estudo em múltiplos casos para que se comprove a generalidade dos resultados. Como este trabalho é o primeiro a verificar a eficácia e eficiência da solução para o estudo de caso no órgão, não há como correlacionar os resultados obtidos a nenhum outro estudo.

& \textbf{Confiabilidade: } Com relação a confiabilidade, \citeonline{yin2001estudo} associa à repetibilidade, desde que seja usada a mesma fonte de dados. Nesse trabalho o protocolo de estudo de caso apresentado nessa seção garantem a repetibilidade desse trabalho e consequentemente a validade relacionada à confiabilidade.

\end{easylist}	


\section{Processo de análise dos dados}

Análise dos dados coletados durante o estudo de caso a ser realizado no TCU? será feita através de 4 etapas:

\begin{easylist}[itemize]	
	
	& \textbf{Categorização: } Organização dos dados em duas categorias - qualitativos e quantitativos. Os 		dados qualitativos referem-se aos questionários realizados. Os dados quantitativos, por sua vez, 			referem-se aos valores numéricos da solução de DW para monitoramento de métricas. 
	& \textbf{Exibição: } Consiste na organização dos dados coletados para serem exibidos através de 				gráficos, tabelas e texto para poderem ser analisados. 
	& \textbf{Verificação: } Atestar padrões, tendências e aspectos específicos dos significados dos 				dados. Procurando assim gerar uma discussão e interpretação de cada dado exibido.
	& \textbf{Conclusão: } Agrupamento dos resultados mais relevantes das discussões e interpretações dos 			dados anteriormente apresentados.
	
	\end{easylist}	


\section{Considerações finais do capítulo}

Esse capítulo teve como objetivo apresentar o protocolo de estudo de caso que será adotado na continuação desse trabalho. A coleta e análise dos dados coletados seguindo esse protocolo ocorrerão no próximo semestre próximo semestre.

\label{estudo de caso}

