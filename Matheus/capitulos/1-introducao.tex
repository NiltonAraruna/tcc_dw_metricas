\chapter{Introdução}

\section{Contexto}

Medir a qualidade do código-fonte  de um software é um processo fundamental no desenvolvimento de um software, pois daí surgem indicadores sobre os efeitos que uma alteração no código irá causar ou sobre os efeitos gerados na qualidade do software após a adesão de uma nova prática na equipe de desenvolvimento \cite{Fenton98}.

O processo de medição, porém, ganha sentido apenas quando há interpretação dos resultados. Usaremos nesse trabalho uma solução desenvolvida por \citeonline{rego_monitoramento_2014} cujo objetivo é atender essa proposta. Para isso, \citeonline{rego_monitoramento_2014} elaborou a seguinte questão de pesquisa: 

\textit{\textbf{Como aumentar a
visibilidade e facilitar interpretação das 
métricas de código-fonte
a fim de apoiar a decisão de refatoração
do ponto de vista de uma equipe de desenvolvimento?}}

\citeonline{rego_monitoramento_2014} desenvolveu um ambiente de Data Warehousing para armazenamento das métricas de código-fonte extraídas do uso da ferramenta de análise  automatizada Analizo buscando atender essa questão de pesquisa. Essa solução tem como objetivo facilitar a interpretação das métricas de código fonte e avaliar indicadores de código limpo no projeto, entre outros objetivos específicos.

\section{Justificativa}

Valores absolutos de métricas não dizem respeito a nada. Solução proposta por Baufaker pode solucionar a interpretação desse valores, automatizando a coleta dos cenários porém

\section{Problema}

Não há uma definiçao concreta a respeito da eficácia e eficiência da solução proposta pelo aluno \citeonline{rego_monitoramento_2014} para monitoramento das métricas de código-fonte no órgão X sendo possível uma substituição do uso do Sonar por essa solução.


\section{Objetivos}

\section{Hipótese ou Metodologia de pesquisa}

\section{Organização do Trabalho}
S
