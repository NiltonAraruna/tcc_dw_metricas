\chapter{Introdução}

\section{Contexto}

Medir a qualidade do código-fonte  de um software é um processo fundamental no desenvolvimento de um software, pois daí surgem indicadores sobre os efeitos que uma alteração no código irá causar ou sobre os efeitos gerados na qualidade do software após a adesão de uma nova prática na equipe de desenvolvimento \cite{Fenton98}.

O processo de medição, porém, ganha sentido apenas quando há interpretação dos resultados.


\section{Justificativa}

Não sei

\section{Problema}

Administração pública não tem capacidade de aferir a qualidade interna dos produtos de software

Baseando-se no problema descrito, foi criada a seguinte questão geral de pesquisa:

\textbf{\textit{Questão de pesquisa }}

\section{Objetivos}

Esse trabalho tem como objetivo principal analisar a eficácia e eficiência do uso de \textit{Data Warehousing} 

\section{Metodologia de pesquisa}

A metodologia adotada:

\section{Organização do Trabalho}

Esse trabalho está dividido em 5 capítulos:

	\begin{easylist}[itemize]	
	
	& \textbf{Capítulo 1 - Introdução:} Esse capítulo tem como objetivo apresentar o contexto que esse trabalho está inserido, o problema sobre o qual ele buscará resolver, qual a justificativa e os objetivos da sua realização e como essa pequisa foi elaborada.
	& \textbf{Capítulo 2 - Métricas de Software:} Capítulo responsável pela explicação teórica a respeito do que são métricas de código e como elas foram utilizadas no desenvolvimento da solução que esse trabalho busca analisar.
	& \textbf{Capítulo 3 - Data Warehouse:} Nesse capítulo serão apresentados conceitos teóricos sobre \textit{Data Warehousing}, assim como a maneira como foi desenvolvido o ambiente de \textit{Data Warehouse} para armazenamento de métricas de código fonte.
	& \textbf{Capítulo 4 - Projeto de estudo de caso:} Será apresentada a estratégia de pesquisa adotada durante o trabalho, buscando elaborar um protocolo para o estudo de caso que será realizado. Elementos de pesquisa como o problema a ser resolvido, os objetivos a serem alcançados no estudo de caso e quais os métodos de coleta e análise dos dados serão identificados e explicados.
	& \textbf{Capítulo 5 - Conclusão:} Além das considerações finais dessa primeira parte do trabalho, serão descritos objetivos para a segunda parte a ser realizada ao término desta.
	
	\end{easylist}	

O capítulo de introdução
