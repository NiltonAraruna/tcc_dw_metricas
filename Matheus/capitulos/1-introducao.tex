\chapter{Introdução}

\section{Contexto}

Medir a qualidade do código-fonte  de um software é um processo fundamental no desenvolvimento de um software, pois daí surgem indicadores sobre os efeitos que uma alteração no código irá causar ou sobre os efeitos gerados na qualidade do software após a adesão de uma nova prática na equipe de desenvolvimento \cite{Fenton98}.

O processo de medição, porém, ganha sentido apenas quando há interpretação dos resultados.


\section{Justificativa}

Não sei

\section{Problema}

Administração pública não tem capacidade de aferir a qualidade interna dos produtos de software

Baseando-se no problema descrito, foi criada a seguinte questão geral de pesquisa:

\textbf{\textit{Questão de pesquisa }}

\section{Objetivos}

Esse trabalho tem como objetivo principal analisar a eficácia e eficiência do uso de \textit{Data Warehousing} 

\section{Metodologia de pesquisa}

A metodologia adotada:

\section{Organização do Trabalho}

Esse trabalho está dividido em 5 capítulos. 
