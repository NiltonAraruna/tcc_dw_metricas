\chapter{Introdução}

\section{Contexto}

Medir a qualidade do código-fonte  de um software é um processo fundamental no desenvolvimento de um software, pois daí surgem indicadores sobre os efeitos que uma alteração no código irá causar ou sobre os efeitos gerados na qualidade do software após a adesão de uma nova prática na equipe de desenvolvimento \cite{Fenton98}.

O processo de medição, porém, ganha sentido apenas quando há interpretação dos resultados e para isso \citeonline{rego_monitoramento_2014} elaborou a seguinte questão de pesquisa: 

\textit{\textbf{Como aumentar a
visibilidade e facilitar interpretação das 
métricas de código-fonte
a fim de apoiar a decisão de refatoração
do ponto de vista de uma equipe de desenvolvimento?}}

Buscando atender essa questão de pesquisa, \citeonline{rego_monitoramento_2014} desenvolveu um ambiente de Data Warehousing para armazenamento das métricas de código-fonte extraídas do uso da ferramenta de análise  automatizada Analizo. Essa solução tem como objetivo facilitar a interpretação das métricas de código fonte e avaliar indicadores de código limpo no projeto, entre outros objetivos específicos.

\section{Justificativa}

\section{Problema}

\section{Questão de Pesquisa}

\section{Objetivos}

\section{Hipótese ou Metodologia de pesquisa}

\section{Organização do Trabalho}
