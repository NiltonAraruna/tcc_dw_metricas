\part{Aspectos Gerais}

\chapter[Aspectos Gerais]{Aspectos Gerais}

Estas instruções apresentam um conjunto mínimo de exigências necessárias a 
uniformidade de apresentação do relatório de Trabalho de Conclusão de Curso 
da FGA. Estilo, concisão e clareza ficam inteiramente sob a 
responsabilidade do(s) aluno(s) autor(es) do relatório.

As disciplinas de Trabalho de Conclusão de Curso (TCC) 01 e Trabalho de 
Conclusão de Curso (TCC) 02 se desenvolvem de acordo com Regulamento 
próprio aprovado pelo Colegiado da FGA. Os alunos matriculados nessas 
disciplinas devem estar plenamente cientes de tal Regulamento. 

\section{Composição e estrutura do trabalho}

A formatação do trabalho como um todo considera três elementos principais: 
(1) pré-textuais, (2) textuais e (3) pós-textuais. Cada um destes, pode se 
subdividir em outros elementos formando a estrutura global do trabalho, 
conforme abaixo (as entradas itálico são \textit{opcionais}; em itálico e
negrito são \textbf{\textit{essenciais}}):

\begin{description}
	\item [Pré-textuais] \

	\begin{itemize}
		\item Capa
		\item Folha de rosto
		\item \textit{Dedicatória}
		\item \textit{Agradecimentos}
		\item \textit{Epígrafe}
		\item Resumo
		\item Abstract
		\item Lista de figuras
		\item Lista de tabelas
		\item Lista de símbolos e
		\item Sumário
	\end{itemize}

	\item [Textuais] \

	\begin{itemize}
		\item \textbf{\textit{Introdução}}
		\item \textbf{\textit{Desenvolvimento}}
		\item \textbf{\textit{Conclusões}}
	\end{itemize}

	\item [Pós-Textuais] \
	
	\begin{itemize}
		\item Referências bibliográficas
		\item \textit{Bibliografia}
		\item Anexos
		\item Contracapa
	\end{itemize}
\end{description}

Os aspectos específicos da formatação de cada uma dessas três partes 
principais do relatório são tratados nos capítulos e seções seguintes.

No modelo \LaTeX, os arquivos correspondentes a estas estruturas que devem
ser editados manualmente estão na pasta \textbf{editáveis}. Os arquivos
da pasta \textbf{fixos} tratam os elementos que não necessitam de 
edição direta, e devem ser deixados como estão na grande maioria dos casos.

\section{Considerações sobre formatação básica do relatório}

A seguir são apresentadas as orientações básicas sobre a formatação do
documento. O modelo \LaTeX\ já configura todas estas opções corretamente,
de modo que para os usuários deste modelo o texto a seguir é meramente
informativo.

\subsection{Tipo de papel, fonte e margens}

Papel - Na confecção do relatório deverá ser empregado papel branco no 
formato padrão A4 (21 cm x 29,7cm), com 75 a 90 g/m2.

Fonte – Deve-se utilizar as fontes Arial ou Times New Roman no tamanho 12 
pra corpo do texto, com variações para tamanho 10 permitidas para a 
wpaginação, legendas e notas de rodapé. Em citações diretas de mais de três 
linhas utilizar a fonte tamanho 10, sem itálicos, negritos ou aspas. Os 
tipos itálicos são usados para nomes científicos e expressões estrangeiras, 
exceto expressões latinas.

Margens - As margens delimitando a região na qual todo o texto deverá estar 
contido serão as seguintes: 

\begin{itemize}
	\item Esquerda: 03 cm;
	\item Direita	: 02 cm;
	\item Superior: 03 cm;
	\item Inferior: 02 cm. 
\end{itemize}

\subsection{Numeração de Páginas}

A contagem sequencial para a numeração de páginas começa a partir da 
primeira folha do trabalho que é a Folha de Rosto, contudo a numeração em 
si só deve ser iniciada a partir da primeira folha dos elementos textuais. 
Assim, as páginas dos elementos pré-textuais contam, mas não são numeradas 
e os números de página aparecem a partir da primeira folha dos elementos 
textuais que é a Introdução. 

Os números devem estar em algarismos arábicos (fonte Times ou Arial 10) no 
canto superior direito da folha, a 02 cm da borda superior, sem traços, 
pontos ou parênteses. 

A paginação de Apêndices e Anexos deve ser contínua, dando seguimento ao 
texto principal.

\subsection{Espaços e alinhamento}

Para a monografia de TCC 01 e 02 o espaço entrelinhas do corpo do texto 
deve ser de 1,5 cm, exceto RESUMO, CITAÇÔES de mais de três linhas, NOTAS 
de rodapé, LEGENDAS e REFERÊNCIAS que devem possuir espaçamento simples. 
Ainda, ao se iniciar a primeira linha de cada novo parágrafo se deve 
tabular a distância de 1,25 cm da margem esquerda.

Quanto aos títulos das seções primárias da monografia, estes devem começar 
na parte superior da folha e separados do texto que o sucede, por um espaço 
de 1,5 cm entrelinhas, assim como os títulos das seções secundárias, 
terciárias. 

A formatação de alinhamento deve ser justificado, de modo que o texto fique 
alinhado uniformemente ao longo das margens esquerda e direita, exceto para 
CITAÇÕES de mais de três linhas que devem ser alinhadas a 04 cm da margem 
esquerda e REFERÊNCIAS que são alinhadas somente à margem esquerda do texto 
diferenciando cada referência.

\subsection{Quebra de Capítulos e Aproveitamento de Páginas}

Cada seção ou capítulo deverá começar numa nova pagina (recomenda-se que 
para texto muito longos o autor divida seu documento em mais de um arquivo 
eletrônico). 

Caso a última pagina de um capitulo tenha apenas um número reduzido de 
linhas (digamos 2 ou 3), verificar a possibilidade de modificar o texto 
(sem prejuízo do conteúdo e obedecendo as normas aqui colocadas) para 
evitar a ocorrência de uma página pouco aproveitada.

Ainda com respeito ao preenchimento das páginas, este deve ser otimizado, 
evitando-se espaços vazios desnecessários. 

Caso as dimensões de uma figura ou tabela impeçam que a mesma seja 
posicionada ao final de uma página, o deslocamento para a página seguinte 
não deve acarretar um vazio na pagina anterior. Para evitar tal ocorrência, 
deve-se re-posicionar os blocos de texto para o preenchimento de vazios. 

Tabelas e figuras devem, sempre que possível, utilizar o espaço disponível 
da página evitando-se a \lq\lq quebra\rq\rq\ da figura ou tabela. 

\section{Cópias}

Nas versões do relatório para revisão da Banca Examinadora em TCC1 e TCC2, 
o aluno deve apresentar na Secretaria da FGA, uma cópia para cada membro da 
Banca Examinadora.

Após a aprovação em TCC2, o aluno deverá obrigatoriamente apresentar a 
versão final de seu trabalho à Secretaria da FGA na seguinte forma:

\begin{description}
	\item 01 cópia encadernada para arquivo na FGA;
	\item 01 cópia não encadernada (folhas avulsas) para arquivo na FGA;
	\item 01 cópia em CD de todos os arquivos empregados no trabalho;
\end{description}

A cópia em CD deve conter, além do texto, todos os arquivos dos quais se 
originaram os gráficos (excel, etc.) e figuras (jpg, bmp, gif, etc.) 
contidos no trabalho. Caso o trabalho tenha gerado códigos fontes e 
arquivos para aplicações especificas (programas em Fortran, C, Matlab, 
etc.) estes deverão também ser gravados em CD. 

O autor deverá certificar a não ocorrência de “vírus” no CD entregue a 
secretaria. 

