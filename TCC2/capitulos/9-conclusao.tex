\chapter{Conclusão}

Neste trabalho, foi apresentado o problema do processo de aferição da qualidade interna dos produtos de \textit{software} desenvolvidos por terceirizadas em órgãos públicos brasileiros. Para realizar uma investigação empírica deste problema, com vistas à proposição de uma solução, foi realizado um estudo do uso de uma solução de \textit{Data Warehouse} para o monitoramento de métricas, cenários de limpeza, bugs e violações de código-fonte no contexto das unidades da  CETEC e GITECBR da CAIXA Econômica Federal.

A solução de DW proposta por \citeonline{rego_monitoramento_2014} foi utilizada, adaptada, melhorada e evoluída neste trabalho com o objetivo de aumentar o valor de aferição da qualidade interna de software monitorando métricas, cenários de limpeza, \textit{bugs} e violações de código-fonte. Também foi construído um \textit{dashboard} gerencial visando dar maior visibilidade das medidas analisadas pela solução.

Antes da solução de DW ser apresentada, foi realizada uma revisão bibliográfica a respeito de qualidade interna de software, de métricas, de \textit{Data Warehouse} e acerca da definição de estudo de caso. Os conceitos levantados na revisão bibliográfica foram apresentados com o objetivo de facilitar na compreensão da solução adotada. Foi apresentada a plataforma da solução, quer sejam: o ambiente e as ferramentas utilizadas na solução de DW. 

Também foi  exibido o projeto de pesquisa do estudo realizado. Conceitos fundamentais do projeto de pesquisa, como qual o problema a ser resolvido e a questão que o caracteriza, foram identificados e apresentados. Em seguida, utilizou-se da  técnica GQM para, através de objetivos, questões específicas e métricas, responder a questão de pesquisa. As etapas pelas quais o estudo de caso atravessou foram descritas e modeladas de forma sequencial, de tal forma a obedecer a revisão bibliográfica sobre o estudo de caso.

Por fim, foi exibido a parte referente a execução do estudo de caso, coleta e análise dos dados obtidos. A execução, coleta e a análise dos dados serviram para evidenciarmos a corretude e a correlação dos dados apresentados pela solução de \textit{Data Warehouse} monitorando as métricas, cenários de limpeza, \textit{bugs} e violações de código-fonte do sistema SIGET adquerido pela CAIXA, com o objetivo de responder a questão de pesquisa definida.


\section{Limitações}


\section{Trabalhos Futuros}