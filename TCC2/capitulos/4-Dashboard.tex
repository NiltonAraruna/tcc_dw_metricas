\chapter{Dashboard}
\label{chap:dashboard}

\section{A Importância do \textit{Dashboard}}

A orientação visual dos \textit{dashboards} é importante devido à velocidade da percepção de que é geralmente necessária para monitorar informações. Quanto mais rápido se deseja avaliar o que está acontecendo, mais deve-se confiar no meio gráfico para mostrar a informação. O texto deve ser lido, o que envolve um processo relativamente lento, sendo que certas propriedades visuais, no entanto, pode ser percebida de relance, sem pensamento consciente.  

O processo de monitorização visual envolve uma série de passos sequenciais que o \textit{dashboard} deve ser concebido para apoio. O usuário ao começar a obter uma visão geral do que está acontecendo deve rapidamente identificar o que precisa de atenção, para que em seguida, o usuário olhe mais de perto cada uma dessas áreas que precisam de atenção para entendê-las bem o suficiente para determinar se algo deve ser feito sobre eles. O monitoramento é uma atividade cognitiva que recebe a entrada principalmente através do canal visual, porque este é o sentido mais poderoso, que é capaz de trabalhar em altas velocidades de entrada, capaz de detectar diferenças sutis e complexas.\cite{few2006}

Segundo \citeonline{few2006}, um \textit{dashboard} é um display visual das informações mais importantes necessárias para alcançar um ou mais objetivos, consolidados e organizados em uma única tela para que a informação possa ser monitorada em um piscar de olhos. \citeonline{few2006} categorizou  diversos tipos de \textit{dashboards} sendo estratégicas, analíticas, ou operacionais, e as características do design no que tange à sugestão de organização variam para dar suporte às necessidades de cada categoria:

\begin{easylist}[itemize]

& fins estratégicos - O uso primários de \textit{dashboards} nos dias de hoje é para propósitos estratégicos, oferecem uma rápida visão que os tomadores de decisão precisam para monitorar a saúde e as oportunidades de um negócio.

& fins analíticos - Mais sofisticação para as mídias de exibição, para que os analistas possam examinar melhor dados complexos e relacionamentos. \textit{Dashboards} analíticos devem suportar interações com os dados, como aprofundamentos em camadas detalhadas, não apenas para ver o que está acontecendo, mas para examinar as causas. 

& fins operacionais - \textit{Dashboards} que monitorem operações devem manter consciência das atividades e eventos que estão mudando constantemente e podem demandar atenção e resposta.

\end{easylist}

Além de categorizar os tipos de \textit{dashboards}, \citeonline{few2006} também evidencia o primordial para se obter um \textit{dashboard} de qualidade:

\begin{easylist}[itemize]

& Disponibilizar a informação diretamente relacionada num único ecrã, ou seja, evitar partir a informação por várias páginas; 

& Evitar a necessidade de \textit{scrolling}; 

& Contextualizar a informação disponibilizada; 

& Incluir fatores de comparação e sugerir ações na visualização dos indicadores; 

& Utilizar escalas adequadas, que devem dar uma perspectiva real das quantidades apresentadas e não podem iludir os utilizadores; 

& Utilizar níveis de precisão adequados nos indicadores, pois evita perdas de tempo com leituras e interpretações de informação desnecessárias e pouco relevantes;

& Escolher os indicadores mais adequados, que facilitem e acelerem a interpretação da informação disponibilizada;

& Escolher soluções gráficas flexíveis e adequadas que facilitem e acelerem a interpretação da informação disponibilizada; 
 
& Uniformizar a leitura ao longo do \textit{dashboard};

& Facilitar a interpretação da informação disponível para acelerar a sua leitura. Por exemplo, evitar cores berrantes, muito próximas, muito apagadas ou um número muito elevado de cores; 

& Apresentar a informação de forma equilibrada, dado que o espaço utilizado num \textit{dashboard} desce de importância do canto superior esquerdo para o canto inferior direito, e por esta razão, a informação que se destaca na visualização deverá ser a mais importante; 

& Os títulos não devem ser mais apelativos que os indicadores; 

& Destacar a informação mais importante e não cair no erro de chamar a atenção para tudo; 

& Aproveitar bem o espaço disponível, ou seja evitar decorações desnecessárias e ainda evitar soluções de pesada implementação para responder a pormenores visuais; 

& Utilizar cores de forma ponderada, ou seja, utilizar cores apelativas apenas para a informação mais importante, podendo utilizar contrastes; 

& Manter as cores para os mesmos indicadores ao longo do \textit{dashboard} ou para o mesmo tipo de indicador associado; 

& Podem ser utilizadas figuras geométricas para além das cores, tais como o círculo, triângulo ou quadrado como forma de ajudar utilizadores que sofram de daltonismo; 

& Criar uma apresentação apelativa, baseando-se no nossa intuição e naquilo que consideramos que a maioria das pessoas aceita e tolera positivamente.

\end{easylist}

\section{Ferramenta para criação de \textit{Dashboards}}

Neste trabalho foi utilizado o CDE (\textit{community dashboard editor}) na versão 14.12.10.1 para a criação dos \textit{dashboards}. O CDE é um \textit{plugin} para o \textit{Pentaho Business Intelligence}.

O CDE permite o desenvolvimento e a implantação de \textit{dashboards} no \textit{Pentaho Business Intelligence}. O CDE nasceu para simplificar a criação e edição de \textit{dashboards} e é uma ferramenta muito poderosa e completa, combinando \textit{front end} com fontes de dados e componentes personalizados de uma forma perfeita. \cite{CDE}

O CDE possui algumas tecnologia por trás dela:

\begin{easylist}[itemize]

& CDF (\textit{Community Dashboard Framework}) - É um \textit{framework} HTML/javascript que permite criar páginas com relatórios, gráficos e tabelas.

& CDA (\textit{Community Dashboard Access}) - São vários componentes que dão acesso à diferentes tipos de fontes de dados.

& CCC (\textit{Community Chart Components}) - É uma biblioteca de gráficos, que possui um poderoso conjunto de ferramentas de visualização livre e de código aberto. O objetivo do CCC é fornecer aos desenvolvedores o caminho para incluir em seus \textit{dashboards} os tipos de gráficos básicos, sem perder o princípio fundamental: a extensibilidade.

\end{easylist}

Para o design do \textit{dashboard}, o CDE oferece três perspectivas:

\begin{easylist}[itemize]

& \textit{Layout} - Para projetar o layout do seu {dashboard} a partir do zero ou de um templete, ao definir o layout é possível aplicar estilos e adicionar elementos HTML para descrever páginas Web, CSS para controlar o estilo do layout, \textit{JavaScript} para adicionar interatividade e \textit{jQuery} para simplificar todas essas tarefas.

& \textit{Components} - Para adicionar e configurar os diferentes componentes que compõem o seu \textit{dashboard}: Componentes Visuais (caixas de texto, tabelas e gráficos, seletores e relatórios), parâmetros que representam os valores que são partilhados pelos componentes e \textit{Scripts}, que permitem personalizar a aparência ou o comportamento de outros componentes.

& \textit{Datasources} - Define os dados usados pelos componentes. \textit{Dashboards} podem ser povoados por uma variedade de fontes como: Bases de dados, cubos do \textit{Mondrian}, metadados do Pentaho , arquivos XML, \textit{ad-hoc datasource} e transformações do \textit{Kettle}.

\end{easylist}


\section{Considerações Finais do Capítulo}

Nesse capítulo foi apresentado as ferramentas para a criação de \textit{Dashboards} utilizadas neste trabalho, bem como a sua importância. O próximo capítulo será responsável por apresentar a importância das contratações e o resumo da Instrução Normativa 04.

%---------------------------------------------------------------------------------------------------------------------%