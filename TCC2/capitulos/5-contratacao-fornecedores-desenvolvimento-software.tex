\chapter{Contratações de Fornecedores de Desenvolvimento de Software}
\label{chap:contratos}

O não cumprimento do disposto na legislação de licitações e contratos pode acarretar riscos para a contratação de Tecnologia da Informação, sendo de grande importância o conhecimento da legislação de licitações e contratos para que se possam ser usados como base em processos de contratação de fornecedores de desenvolvimento de software.
Assim, neste capítulo será apresentada uma visão sobre a importância da contratação de
serviços de TI e de forma resumida, os conceitos de contratação de serviços de TI presentes na Instrução Normativa Nº 04.

\section{Importância da Contratação de Fornecedores de Desenvolvimento de Software}

De acordo com a \citeonline{ISO:12207}, contrato é o cordo realizado entre duas partes, respaldado pela lei, ou acordo interno similar restrito a uma organização, para o fornecimento de serviços de software ou para o fornecimento, desenvolvimento, produção, operação ou manutenção de um produto de software.

Embora a contratação de serviços de TI tenha papel importante na estratégia organizacional, há muitos riscos que podem frustrar seus resultados. A definição e institucionalização de processos de contratação de serviços de TI, especialmente aqueles relacionados a software, envolvem ações complexas, principalmente no que diz respeito à identificação dos requisitos necessários, a garantia da qualidade dos resultados esperados, os critérios de aceitação, a gestão de mudanças, as transferências de conhecimentos, a legislação pertinente, entre outros. E envolvem também questões de relacionamento entre clientes e fornecedores, o que implica em competências administrativas e jurídicas. Essas complexidades apresentam riscos para as partes envolvidas e, como consequência, é comum a ocorrência de sérios conflitos. Normas e modelos de referência podem ser úteis para resolver esses conflitos \cite{processoContratacao}.

Neste trabalho, a aquisição de \textit{software} é o assunto mais relevante dentro do contexto de contratos,  mais especificamente da atividade de aceitação pelo cliente. A \cite{ISO:12207} dividi a aquisição de \textit{software} e serviços correlatos em quatro atividades, conforme a Figura \ref{atividadesAquisicao}.


\begin{figure}[h!]
\centering
\includegraphics[keepaspectratio=false,scale=0.3]{figuras/figuras_nilton/atividadesAquisicao.eps}
\caption{Atividades de aquisição extraída de \citeonline{ISO:12207}}.
\label{atividadesAquisicao}
\end{figure}

Segundo a \citeonline{ISO:12207},  o propósito da atividade de aceitação pelo cliente é aprovar o \textit{software} e os serviços correlatos (S\&SC) entregues pelo fornecedor quando todos os critérios de aceitação estiverem satisfeitos. Nesta atividade são refinados os critérios de aceitação que foram definidos no plano de projeto e incorporados no pedido de proposta e no contrato. As avaliações podem ser conduzidas no decorrer do contrato, por uma abordagem envolvendo múltiplas iterações e entregas de produtos, ou por meio de uma entrega única. Os S\&SC entregues são analisados para identificar a conformidade aos critérios estabelecidos. As tarefas de avaliação são concebidas de modo a reduzir a interferência com as avaliações executadas pelo fornecedor e a duplicação de esforços de avaliação. Não havendo aprovação do S\&SC, e dependendo das cláusulas contratuais, podem ser planejados e implementados ajustes para que o produto seja submetido a uma nova avaliação. Este ciclo ocorre enquanto o produto não é aprovado, ou até que seja definitivamente rejeitado. As tarefas previstas compreendem:

\begin{easylist}[itemize]
& \textbf{Definir critérios de aceitação}  

& \textbf{Avaliar o produto entregue} 

& \textbf{Manter conformidade com o contrato} 

& \textbf{Aceitar o S\&SC} 
\end{easylist}


A Instrução Normativa Nº 04 trata do processo de contratação de serviços de TI pela Administração Pública Federal direta, autárquica e fundacional. Esta Instrução Normativa IN (Instrução Normativa) será explicada na próxima seção, onde os artigos mais relevantes no contexto da aquisição de \textit{software} e da aceitação de \textit{software} pelo cliente   serão abordados.
%---------------------------------------------------------------------------------------------------------------------%

\section{Instrução Normativa Nº 04}

Instruções normativas são atos expedidos por autoridades administrativas, normas complementares das leis, dos tratados e das convenções internacionais e dos decretos, e não podem transpor, inovar ou modificar o texto da norma que complementam. As instruções normativas visam regulamentar ou implementar o que está previsto nas leis que, no caso do Brasil, são apreciadas, elaboradas e aprovadas pelo Congresso Nacional, e sancionadas pelo Presidente da República.

A \citeonline{Normativa4} dispõe sobre o processo de contratação de Soluções de Tecnologia da Informação pelos órgãos integrantes do Sistema de Administração de Recursos de Tecnologia da Informação e Informática (SISP) do Poder Executivo Federal. Esta IN é a consolidação de um conjunto de boas práticas para Contratação de Solução de TI, que formam o Modelo de Contratações de Soluções de TI (MCTI).

A Instrução Normativa nº 04 \citeonline{Normativa4} está dividida em três capítulos:

\begin{easylist}[itemize]
& \textbf{Capítulo 1:} Diz respeito às disposições gerais.
& \textbf{Capítulo 2:} Diz respeito ao processo de contratação e é dividido em 3 seções.
					
					&& \textbf{Seção 1:} Diz respeito ao Planejamento da Contratação e é  								dividido em 4 Subseções.
					&& \textbf{Seção 2:} Diz respeito à Seleção do Fornecedor.
					&& \textbf{Seção 3:} Diz respeito à Gestão do Contrato e é  								dividido em 4 Subseções.

& \textbf{Capítulo 3:} Apresenta as Disposições Finais.
\end{easylist}

Os artigos mais relevantes e que serão abordados neste trabalho são os artigos 20 e 34, ambos inseridos no capítulo 2 da referida normativa. 

O caput do artigo 20 do \citeonline{Normativa4} trata sobre o modelo de gestão do contrato, definido a partir do Modelo de Execução do Contrato, que deverá contemplar as condições para gestão e fiscalização do contrato de fornecimento da Solução de Tecnologia da Informação. 

O inciso II versa sobre os  procedimentos de teste e inspeção, para fins de elaboração dos Termos de Recebimento Provisório e Definitivo. Ademais, na alínea "a", itens 1,2 e 5 tratam sobre a metodologia, formas de avaliação da qualidade e adequação da solução de Tecnologia da Informação às especificações funcionais e tecnológicas, observando:

\begin{easylist}[itemize]
& \textbf{Item 1:}  Definição de mecanismos de inspeção e avaliação da Solução, a exemplo de inspeção por
amostragem ou total do fornecimento de bens ou da prestação de serviços.

& \textbf{Item 2:} Adoção de ferramentas, computacionais ou não, para implantação e acompanhamento
dos indicadores estabelecidos.

& \textbf{Item 5:} Garantia de inspeções e diligências, quando aplicáveis, e suas formas de exercício.
\end{easylist}

O inciso III versa sobre a fixação dos valores e procedimentos para retenção ou glosa no pagamento, sem
prejuízo das sanções cabíveis, que só deverá ocorrer quando a contratada incindir nas alíneas "a" e "b" do mencionado inciso. 

\begin{easylist}[itemize]
& \textbf{alíneas "a":} não atingir os valores mínimos aceitáveis fixados nos Critérios de Aceitação, não
produzir os resultados ou deixar de executar as atividades contratadas; ou

& \textbf{alíneas "b":} deixar de utilizar materiais e recursos humanos exigidos para fornecimento da solução de tecnologia da informação, ou utilizá-los com qualidade ou quantidade inferior à demandada.

\end{easylist}

Por fim, no inciso IV trata sobre a definição clara e detalhada das sanções administrativas observadas, principalmente, nas alíneas "b" e "c".

\begin{easylist}[itemize]
& \textbf{alíneas "b":} proporcionalidade das sanções previstas ao grau do prejuízo causado pelo descumprimento das respectivas obrigações.

& \textbf{alíneas "c":} as situações em que advertências ou multas serão aplicadas, com seus percentuais
correspondentes, que obedecerão a uma escala gradual para as sanções recorrentes.

\end{easylist} 

Outrossim o artigo 34 da referida normativa trata sobre o monitoramento da execução deverá observar o disposto no Plano de Fiscalização da contratada e o disposto no modelo de gestão do contrato, observando os seguintes incisos:

\begin{easylist}[itemize]
& \textbf{II:} Avaliação da qualidade dos serviços realizados ou dos bens entregues e justificativas, a
partir da aplicação das Listas de Verificação e de acordo com os Critérios de Aceitação definidos em contrato, a cargo dos Fiscais Técnico e Requisitante do Contrato.

& \textbf{III:} Identificação de não conformidade com os termos contratuais, a cargo dos Fiscais
Técnico e Requisitante do Contrato. 

& \textbf{VI:} Encaminhamento das demandas de correção à contratada, a cargo do Gestor do
Contrato ou, por delegação de competência, do Fiscal Técnico do Contrato.

& \textbf{VII:} Encaminhamento de indicação de glosas e sanções por parte do Gestor do Contrato
para a Área Administrativa. 

\end{easylist}
%---------------------------------------------------------------------------------------------------------------------%

%\section{Contrato do Órgão X}



%--------------------------------------------
