\begin{resumo}[Abstract]
 \begin{otherlanguage*}{english}
 
The quality of the software depends on the quality of the source code, a good source code is a good indicator of the software internal quality. Therefore, the monitoring of metrics of a software source-code results in improving its quality. There are several solutions and tools to achieve software monitoring of metrics that can easily extract values of metrics of code. However, the decision regarding what to do with the extracted data is still an issue related to the visibility and interpretation of this data. The purpose of this project is to analyse of the use of the environment Data Warehousing to facilitate interpretation, visibility and evaluation of bugs, violation and source code metrics, associating them with cleansing scenarios. The cleansing scenarios are to support decision making processes that reflect on the alteration of the source code and a software with greater quality. For a better understanding of the DW solution proposedand of the elements that concern its arquitecture and its business requirements, the necessary theoretical fundaments have been presented. The project aims to answer qualitative and quantitative questions about the power about the power of
assistance to the solution of quality benchmarking process cited in CAIXA Econômica Federal.
 
\textbf{Key-words:} Source Code Metrics,  Data Warehousing, Data Warehouse, Effectiveness, 	 efficiency

 \end{otherlanguage*}
\end{resumo}