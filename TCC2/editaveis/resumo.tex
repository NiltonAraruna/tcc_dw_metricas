\begin{resumo}




A qualidade do software depende da qualidade do código-fonte, um bom código-fonte é um bom indicador de
qualidade interna do produto de \textit{software}. Portanto, o monitoramento de métricas de código-fonte de um \textit{software} significa melhorar a sua qualidade. Existem diversas soluções e ferramentas para se obter um monitoramento de métricas de \textit{software} e que conseguem extrair valores de métricas de código com facilidade. Porém, a decisão sobre o que fazer com os dados extraídos ainda é uma
dificuldade relacionada à visibilidade e interpretação dos dados. Este trabalho se propõe a analisar o uso de um ambiente de \textit{Data Warehousing(DW)} para facilitar a interpretação, visibilidade e avaliação dos \textit{bugs}, das violações e das métricas de código-fonte, associando-as a cenários de limpeza. Os cenários de limpeza buscam apoiar as tomadas de decisão que reflitam na alteração do código-fonte e um \textit{software} com mais qualidade. Para um melhor entendimento da solução DW proposta e dos elementos que dizem respeito a sua arquitetura e seus requisitos de negócio foram apresentas as fundamentações teóricas necessárias. Um projeto para a realização de uma investigação empírica foi elaborado utilizando a técnica de estudo de caso. O projeto visa responder questões qualitativas e quantitativas a respeito do poder de assistência ao processo de aferição da qualidade da solução citada na CAIXA Econômica Federal.
 
 

 \vspace{\onelineskip}
    
 \noindent
 \textbf{Palavras-chaves}: Métricas de Código-Fonte. \textit{Data Warehousing}. \textit{Data Warehouse}.
\end{resumo}
