\begin{resumo}

Monitorar métricas de código fonte de um software significa monitorar sua qualidade interna. Embora seja possível extrair valores de métricas de código com facilidade através de ferramentas já existentes, a decisão sobre o que fazer com os dados extraídos ainda esbarra na dificuldade relacionada à visualização e interpretação dos dados. Nesse contexto, este trabalho busca analisar a eficácia e eficiência do uso de um ambiente de \textit{Data Warehousing} para facilitar a interpretação das métricas de código fonte, associando-as a cenários de limpeza com o objetivo de apoiar as tomadas de decisão a respeito de mudanças no código. Este trabalho apresenta as fundamentações teóricas necessárias para o entendimento dessa solução bem como elementos que dizem respeito a sua arquitetura e requisitos de negócio. Para realizar a pesquisa sobre sua eficácia e eficiência, foi elaborada uma investigação empírica através da técnica do estudo de caso, que visa responder questões qualitativas e quantitativas a respeito do uso desse ambiente no TCU.

 \vspace{\onelineskip}
    
 \noindent
 \textbf{Palavras-chaves}: Métricas de Código-Fonte. \textit{Data Warehousing}. \textit{Data Warehouse}
\end{resumo}

