\begin{resumo}[Abstract]
 \begin{otherlanguage*}{english}

Monitor metrics of source code of a software means to monitor its internal quality. Although it is possible to extract values from code metrics with ease through existing tools, the decision about what to do with the extracted data still faces the difficulty related to the visualization and interpretation of data. In this context, this paper seeks to analyze th effectiveness and efficiency of the use of and environment of \textit{Data Warehousing} to facilitate the interpretation of source code metrics, linking them to cleasing scenarios with the aim of supporting decision-making at regarding changes in the code. This paper presents the theoretical framework necessary for understanding this solution as well as elements that relate to their architecture and business requirements. To conduct research on their effectiveness and efficiency, an empirical research has been prepared by the technique of case study that aims to answer qualitative and quantitative questions regarding the use of this environment on the TCU.
 
\textbf{Key-words:} Source Code Metrics,  Data Warehousing, Data Warehouse

 \end{otherlanguage*}
\end{resumo}



 		