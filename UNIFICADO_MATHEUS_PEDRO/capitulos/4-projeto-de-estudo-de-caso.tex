\chapter{Projeto de Estudo de Caso}

Este capítulo irá apresentar a estratégia de pesquisa adotada durante o trabalho, buscando elaborar um protocolo para o estudo de caso que será realizado. Elementos de pesquisa como o problema a ser resolvido e quais são os objetivos a serem alcançados serão identificados e explicados. Também será apresentado o método para coleta dos dados e como eles serão analisados.


\section{Definição Sobre Estudo de Caso}

O estudo de caso é uma estratégia de pesquisa utilizada para investigar um tópico de manira empírica através de um conjunto de procedimentos pré-especificados \cite{yin2001estudo}. Buscando diferenciar o estudo de caso de outras estratégias de pesquisa, \citeonline{yin2001estudo} esclarece que um estudo de caso deve focalizar acontecimentos contemporâneos, não havendo assim exigência quanto ao controle sobre os eventos comportamentais. Dessa forma, o estudo de caso difere de um experimento pelo motivo que neste há controle e manipulação sobre os eventos, diferentemente do estudo de caso, que não os manipula. Em suma, a essência de qualquer estudo de caso reside em esclarecer uma decisão ou um conjunto de decisões, considerando o motivo pelo qual elas foram tomadas e qual os resultados das suas implementações \cite{schramm_notes_1971}. 

Uma vantagem dos estudos de caso é que eles são mais fáceis de planejar e são mais realistas, mas as desvantagens são que os resultados são difíceis de generalizar e mais difícil de interpretar , ou seja, é possível mostrar os efeitos de uma situação típica, mas requer mais análise para que se possa generalizar a outras situações \cite{wohlin2012experimentation}.


A estrutura do protocolo de estudo de caso proposta se baseia em \citeonline{case-study-template-2008} e se apresenta dividida nos seguintes tópicos, cada um com seu objetivo específico:

\begin{easylist}[itemize]

& \textbf{Background - Seção \ref{sec:Background}}: Identificar outros estudos acerca do tópico, definir a questão de pesquisa principal e suas proposições derivadas que serão abordado por este estudo.

& \textbf{Design - Seção \ref{sec:Design}}: Identificar se o projeto de pesquisa é um caso único ou múltiplo bem como seu propósito geral.

& \textbf{Seleção - Seção \ref{sec:Seleção}}: Apresentar critérios para a seleção do caso e descrição do objeto de estudo a ser analisado.

& \textbf{Fonte e Método de Coleta de Dados - Seção \ref{sec:Fonte e Método de Coleta de Dados}}: Identificar os dados que serão coletados, definindo um plano para a coleta e como a informação será armazenada.

& \textbf{Processo de Análise dos Dados - Seção \ref{sec:Análise}}: Identificar os critérios para interpretação dos resultados do estudo de caso, relacionar os dados com a questão de pesquisa e elaborar a explicação do encontrado.

& \textbf{Ameaças a validade do estudo de caso - Seção \ref{sec:Validade}}: Elicitar tipos de validades aplicáveis a um estudo de caso baseando-se no trabalho desenvolvido por \citeonline{yin2001estudo}, sendo elas: constructo, interna, externa e confiabiliade.

\end{easylist}


\section{Background}\label{sec:Background}

Esta seção contém referências sobre os trabalhos que antecederam esse estudo de caso dentro de um contexto similar ao que foi apresentado, assim como a própria questão geral de pesquisa a ser respondida com todos os elementos necessários para respondê-la.

\subsection{Trabalhos Antecedentes}

O principal trabalho que antecede essa ideia foi desenvolvido por \citeonline{rego_monitoramento_2014}, no qual a solução para monitoramento de métricas de código fonte utilizando \textit{Data Warehouse} foi desenvolvida. \citeonline{novello_uma_2006} utilizou \textit{Data Warehouse} para monitoramento de métricas no processo de desenvolvimento de software, porém não essas métricas não eram de código fonte. Não foram encontrados outros trabalhos que usassem \textit{Data Warehouse} para aferir qualidade do que está sendo desenvolvido relacionando métricas de código fonte a cenários de limpeza, tal qual foi feito por \citeonline{rego_monitoramento_2014}.

\subsection{Questão de Pesquisa}

Toda pesquisa se inicia com algum tipo de problema, porém nem todo problema é passível de tratamento científico. Para que um problema seja de natureza científica ele deve estar envolvido a variáveis que podem ser testadas \cite{gil_como_2002}. Por causa disso, \citeonline{gil_como_2002} ainda define que todo problema deve ser empírico e suscetível de solução, além de ser delimitado a uma dimensão viável.

A questão de pesquisa deriva do problema identificado e busca definir o propósito da medição e o objeto que será medido, seja ele produto, processo ou recurso. Além disso deve caracterizar o assunto a ser tratado e o ponto de vista que se delimita o escopo \cite{Basili96b} \cite{caldiera_goal_1994}.

Para atender a questão de pesquisa foi utilizado o mecanismo goal-question-metrics (GQM). O GQM combina em si muitas das técnicas de medição e as generaliza para incorporar processos, produtos ou recursos, o que torna seu uso adaptável a ambientes diferentes \cite{caldiera_goal_1994}. O GQM possui objetivos específicos para a questão de pesquisa, cada um contendo questões específicas que buscam responder através da coleta de métricas a questão geral de pesquisa e assim atender o problema do estudo de caso \cite{Basili96b}. A estrutura do estudo de caso englobando o GQM utilizado é apresentada da seguinte maneira:  

\begin{figure}[h!]
\centering
\includegraphics[keepaspectratio=false,scale=0.40]{figuras/figuras_matheus/projeto_de_pesquisa.eps}
\caption{Estrutura do estudo de caso}
\label{fig:pesquisa}
\end{figure}
\FloatBarrier

Aplicando o modelo estrutural adotado e levando em conta a análise do problema identificado na seção \ref{intro_problema} foi elaborada a seguinte questão de pesquisa:

\textbf{Questão de Pesquisa: \textit{O uso de um ambiente de Data Warehousing para aferição e monitoramento da qualidade interna do código-fonte é eficaz e eficiente do ponto de vista da equipe de qualidade?}}

Os objetivos identificados para responder a questão geral de pesquisa, bem como suas questões específicas e as métricas levantadas para cada questão serão apresentadas a seguir.

\textbf{Objetivo 01:} Avaliar a eficácia e eficiência do uso de \textit{Data Warehouse} para monitoramento de métricas de código fonte. \newline

% Questão 01

\textbf{QE01:} Quantas tomadas de decisão foram realizadas pela equipe baseando-se no uso da solução desenvolvida em um <período de tempo>?

\textbf{Fonte:} Registro de observação em campo.

\textbf{Métrica:} Número de decisões tomadas/tempo. \newline

% Questão 02

\textbf{QE02: } Quantas tomadas de decisão ao todo foram realizadas pela equipe baseando-se no uso da solução desenvolvida?

\textbf{Fonte:} Registro de observação em campo.

\textbf{Métrica:} Número de decisões tomadas. \newline


% Questão 03

\textbf{QE03: } Qual a avaliação da equipe de qualidade quanto a detecção de cenários de limpeza de código?

\textbf{Fonte:} Questionário com equipe de qualidade.

\textbf{Métrica:} muito bom, bom, regular, ruim, muito ruim. \newline

% Questão 04

\textbf{QE04: } Com que frequência a equipe de qualidade encontra falhas relacionados à utilização da ferramenta em um determinado intervalo de tempo?

\textbf{Fonte:} Registro de observação em campo.

\textbf{Métrica:} Quantidade de falha / tempo (release, sprint).

\textbf{Interpretação da métrica:} Quanto mais próximo de zero melhor 0=<X \newline

% Questão 05

\textbf{QE05: } Qual a quantidade total de falhas encontradas pela equipe de qualidade relacionadas à utilização da ferramenta?

\textbf{Fonte:} Registro de observação em campo (áudio/vídeo).

\textbf{Métrica:} Quantidade de falhas. \newline

% Questão 06

\textbf{QE06: } Qual a proporção do uso da ferramenta para tomada de decisões?

\textbf{Fonte:} Questionário com equipe de qualidade.

\textbf{Métrica:} Número de decisões tomadas / número de vezes que a solução foi usada. \newline


% Questão 07

\textbf{QE07: } Qual a quantidade de cenários que foram corrigidos após utilização da solução?

\textbf{Fonte:} Código fonte.

\textbf{Métrica:} Números de cenários corrigidos / número de cenários encontrados. \newline

% Questão 08

\textbf{QE08: } Qual o nível de satisfação do uso da solução em comparação à solução anterior? 

\textbf{Fonte:} Equipe de qualidade.

\textbf{Métrica:} Muito satisfeito, Satisfeito, Neutro, Insatisfeito, Muito Insatisfeito. \newline

% Questão 09

\textbf{QE09: } Qual a taxa de oportunidade de melhoria de código em um intervalo de tempo (sprint, release)? 

\textbf{Fonte:} Código fonte.

\textbf{Métrica:} Taxa de oportunidade de melhoria de código: $$ T_r =   \frac{{\sum_{i=1}^{n}{Ce_i}}}{\sum_{i=1}^{n}{Cl_i}} $$ $ Ce $ é o total de cenários de limpezas identificados e $ Cl $  é o total de classes em um intervalo de tempo (sprint, release).

\textbf{Interpretação da métrica:} O intervalo qualitativo que definirá a interpretação desta será desenvolvido a partir de uma análise estatística a ser realizada na segunda parte desse trabalho, analisando valores coletados na aferição dessa métrica em projetos de software livre previamente selecionados para servirem de referência. 

\section{Design} \label{sec:Design}

\citeonline{stake_art_1995} identifica três modalidades de estudo de caso: intrínseco, instrumental e coletivo. Dentre essas definições, este estudo de caso se caracteriza como instrumental, pois se busca um entendimento geral sobre o problema de pesquisa a partir do estudo de um caso particular. Tal característica se assemelha ao tipo de estudo de caso definido como exploratório por \citeonline{yin2001estudo}, onde a visão acurada de um caso particular poderá fornecer uma visão geral do problema considerado.

A justificativa para este estudo de um caso particular se dá pelo motivo de que o tema abordado foi pouco explorado até o momento. Assim espera-se que este caso único componha um estudo de múltiplos casos, pois segundo \citeonline{gil_como_2002} esta abordagem pode inserir as evidências coletadas nesse trabalho em diferentes contextos, com o intuito de elaborar uma pesquisa de melhor qualidade.

\section{Seleção} \label{sec:Seleção}

FALAR DO ERIC e do TCU

\section{Fonte dos Dados Coletados e Método de Coleta} \label{sec:Fonte e Método de Coleta de Dados}

Nesse estudo de caso, os dados foram coletados através de registros de observações em campo, questionários e resultados gerados pela própria solução após análise direta do código fonte.

Os registros oriundos de observações em campo são coletados durante encontros realizados no próprio TCU com a equipe responsável pela tomada de decisões de qualidade de código fonte. Nesses encontros, a solução proposta é utilizada e qualquer atitude relacionada ao seu uso pela equipe é registrado.

A adoção de questionários foi utilizada tanto para dados qualitativos quanto para dados quantitativos. Um exemplo disso é a questão específica 06, em que se busca saber a proporção de decisões tomadas influenciadas pela solução via questionário com a empresa. O uso de questionário na questão 06 é quantitativo, enquanto na questão 08 é feito um questionário para saber o nível de satisfação da empresa quanto ao uso da solução, sendo esse um tipo de dado qualitativo.

Dados resultantes da própria ferramenta, como gráficos automaticamente gerados, serão coletados a medida que a ferramenta for utilizada ao longo do tempo.

\section{Processo de Análise dos Dados} \label{sec:Análise}

Análise dos dados coletados durante o estudo de caso a ser realizado no TCU será realizado através de 4 etapas:

\begin{easylist}[itemize]	
	
	& \textbf{Categorização: } Organização dos dados em duas categorias - qualitativos e quantitativos. Os 		dados qualitativos referem-se aos questionários realizados. Os dados quantitativos, por sua vez, 			referem-se aos valores numéricos da solução de DW para monitoramento de métricas. 
	& \textbf{Exibição: } Consiste na organização dos dados coletados para serem exibidos através de 				gráficos, tabelas e texto para poderem ser analisados. 
	& \textbf{Verificação: } Atestar padrões, tendências e aspectos específicos dos significados dos 				dados. Procurando assim gerar uma discussão e interpretação de cada dado exibido.
	& \textbf{Conclusão: } Agrupamento dos resultados mais relevantes das discussões e interpretações dos 			dados anteriormente apresentados.
	
	\end{easylist}

\section{Ameaças a Validade do Estudo de Caso} \label{sec:Validade}

\citeonline{yin2001estudo} descreve como principais ameaças relacionadas à validade do estudo de caso as ameaças relacionadas à validade do constructo, à validade interna, à validade externa e à confiabilidade. As quatro ameaças definidas por ele, bem como a forma usada nesse trabalho para preveni-las, são descritas da seguinte maneira: 

\begin{easylist}[itemize]	

& \textbf{Validade do Constructo: } A validade de construção está presente na fase de coleta de dados, quando deve ser evidenciado as múltiplas fontes de evidência e a coleta de um conjunto de métricas para que se possa saber exatamente o que medir e quais dados são relevantes para o estudo, de forma a responder as questões de pesquisa \cite{yin2001estudo}. Buscou-se garantir a validade de construção ao definir objetivos com evidências diferentes. Estas, por sua vez, estão diretamente relacionadas com os objetivos do estudo de caso e os objetivos do trabalho. 

& \textbf{Validade interna: } Para \citeonline{yin2001estudo} o uso de várias fontes de dados e métodos de coleta permite a triangulação, uma técnica para confirmar se os resultados de diversas fontes e de diversos métodos convergem. Dessa forma é possível aumentar a validade interna do estudo e aumentar a força das conclusões.
A triangulação de dados se deu pelo resultado da solução de \textit{Data Warehouse} que utiliza o código-fonte (explicada no capítulo \ref{chap:arquitetura} ) , pela análise de questionários e pelos dados coletados através de entrevistas.

& \textbf{Validade externa: } Por este ser um caso único, a generalização do estudo de caso se dá de maneira pobre \cite{yin2001estudo}. Assim é necessária a utilização do estudo em múltiplos casos para que se comprove a generalidade dos resultados. Como este trabalho é o primeiro a verificar a eficácia e eficiência da solução para o estudo de caso no TCU, não há como correlacionar os resultados obtidos a nenhum outro estudo.

& \textbf{Confiabilidade: } Com relação a confiabilidade, \citeonline{yin2001estudo} associa à repetibilidade, desde que seja usada a mesma fonte de dados. Nesse trabalho o protocolo de estudo de caso apresentado nessa seção garantem a repetibilidade desse trabalho e consequentemente a validade relacionada à confiabilidade.

\end{easylist}	


\section{Considerações Finais do Capítulo} 

Esse capítulo teve como objetivo apresentar o protocolo de estudo de caso que será adotado na continuação desse trabalho. A coleta e análise dos dados coletados seguindo esse protocolo ocorrerão no próximo semestre próximo semestre.

\label{estudo de caso}

