\chapter{Conclusões e próximos passos}

Neste trabalho foi apresentado o problema do processo de aferição da qualidade interna dos produtos de \textit{software} desenvolvidos por terceirizadas em órgãos públicos brasileiros. Para realizar uma investigação empírica deste problema, com vistas a proposição de uma solução, será feito um estudo do uso de uma solução de \textit{Data Warehouse} para monitoramento de métricas de código-fonte, no contexto da unidade CETEC/CAIXA.

A solução de DW proposta por \citeonline{rego_monitoramento_2014} foi adotada neste trabalho com o objetivo de evidenciar a sua eficácia e eficiência no monitoramento de métricas de código-fonte para assistir ao processo de aferição de qualidade interna dos produtos de software desenvolvido por terceirizadas, do ponto de vista da equipe de qualidade da CAIXA. 

Antes da solução de DW ser apresentada foi realizada uma revisão bibliográfica a respeito de qualidade interna de software, de métricas, de \textit{Data Warehouse} e acerca da definição de estudo de caso. Os conceitos levantados na revisão bibliográfica foram apresentados com o objetivo de facilitar na compreensão da solução adotada. Foi apresentada a plataforma da solução, quer sejam: o ambiente e as ferramentas utilizadas na solução de DW adotada. 

Por fim foi  exibido o projeto de pesquisa a ser realizada. Conceitos fundamentais do projeto de pesquisa como qual o problema a ser resolvido e a questão que o caracteriza foram identificados e apresentados. Em seguida, utilizou-se da  técnica GQM para, através de objetivos, questões específicas e métricas, responder a questão de pesquisa. As etapas pelas quais o estudo de caso atravessará foram descritas e modeladas de forma sequencial, de tal forma a obedecer a revisão bibliográfica sobre o estudo de caso.

A primeira parte do trabalho de conclusão de curso teve como foco o planejamento do estudo de caso. A próxima etapa será responsável pela coleta e análise dos dados a serem obtidos. A coleta e a análise dos dados servirão para evidenciarmos a eficácia e a eficiência do uso de \textit{Data Warehouse} no monitoramento de métricas de código-fonte em um órgão público, com o objetivo de responder a questão de pesquisa definida.

Na segunda parte desse trabalho também serão definidos mais cenários de limpeza relacionando-os a um conjunto de métricas de código. Também será construído um \textit{dashboard} gerencial visando dar maior visibilidade das medidas analisadas, em particular, procurando verificar como a solução proposta poderia ser utilizada como suporte ao processo de aferição da qualidade interna, hoje realizado na organização analisada.