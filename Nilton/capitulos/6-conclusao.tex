\chapter{Conclusão}

Neste trabalho foi apresentado o problema do processo de aferição da qualidade interna dos produtos de \textit{software} desenvolvidos por terceirizadas em órgãos públicos brasileiros. Para tentar suprimir este problema foi feito um estudo sobre uma solução de \textit{Data Warehouse} para monitoração de métrica de código-fonte.

A solução de DW proposta por \citeonline{rego_monitoramento_2014} foi adotada neste trabalho com o objetivo de evidenciar a sua eficácia e eficiência no monitoramento de métricas de código-fonte para assistir ao processo de aferição de qualidade interna dos produtos de software desenvolvido por terceirizadas, do ponto de vista da equipe de qualidade de um órgão público. 

Antes da solução de DW ser apresentada foi realizada uma revisão bibliográfica a respeito de qualidade interna de software, de métricas, de \textit{Data Warehouse} e acerca da definição de estudo de caso. Os conceitos levantados na revisão bibliográfica foram apresentados com o objetivo de facilitar na compreensão da solução adotada. Somente após  a apresentação dos conceitos foi exposta a metodologia, o ambiente e as ferramentas utilizadas na solução de DW adotada. 

Por fim foi  exibido o projeto de pesquisa a ser realizada. Conceitos fundamentais do projeto de pesquisa como qual o problema a ser resolvido e a questão que o caracteriza foram identificados e apresentados. Em seguida, utilizou-se da  técnica GQM para, através de objetivos, questões específicas e métricas, responder a questão de pesquisa. As etapas pelas quais o estudo de caso atravessará foram descritas e modeladas de forma sequencial, de tal forma a obedecer a revisão bibliográfica sobre o estudo de caso.

A primeiro parte do trabalho de conclusão de curso teve como foco o planejamento do estudo de caso, a próxima etapa será responsável pela coleta e análise dos dados a serem obtidos. A coleta e a análise dos dados servirão para evidenciarmos a eficácia e a eficiência do uso de \textit{Data Warehouse} no monitoramento de métricas de código-fonte em um órgão público, com o objetivo de responder a questão de pesquisa definida.

Na segunda parte desse trabalho também serão criados mais cenários de limpeza relacionando-os a um conjunto de métricas de código e será realizada uma análise estatística da taxa de oportunidade de melhoria de código em projetos de \textit{software} livre. O intervalo qualitativo para a taxa de oportunidade pode ser obtido através da análise estatística, um intervalo qualitativo possibilita que a eficácia da solução adotada possa ser aferida com mais facilidade e exatidão no estudo de caso.   