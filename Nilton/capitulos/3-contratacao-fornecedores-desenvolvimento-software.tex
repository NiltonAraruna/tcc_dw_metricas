\chapter{Contratações de Fornecedores de Desenvolvimento de Software}
\label{chap:contratos}

O descumprimento da legislação de licitações e contratos gera riscos para a contratação de tecnologia da informação e, portanto, devem ser conhecidos e usados como
base de qualquer processo de contratação de fornecedores de desenvolvimento de software.
Assim, neste capítulo será apresentada uma visão sobre a importância da contratação de
serviços de TI e a caracterização de contratação de tecnologia da informação pelas organizações públicas brasileiras, segundo a legislação pertinente, apresentando-se, de forma resumida, os conceitos de contratação de serviços de TI presentes em normas, modelos, guias e processos de contratação de soluções de TI.

\section{Importância da Contratação de Fornecedores de Desenvolvimento de Software}

Um dos atos administrativos principais, mais complexos e mais frequentemente
utilizados é a contratação. Contratar é fazer contrato, que é um acordo ou convenção
entre duas ou mais pessoas para a execução de alguma coisa, sob determinadas condições.
O contrato é, portanto, o documento em que se registra esse acordo ou convenção (MPOG,
2011).

A definição e institucionalização de processos de contratação de serviços de TI,
especialmente aqueles relacionados a software, envolvem ações complexas, principalmente
no que diz respeito à identificação dos requisitos necessários, a garantia da qualidade dos resultados esperados, aos critérios de aceitação, a gestão de mudanças, as transferências de conhecimentos, a legislação pertinente, entre outros. Envolvem também questões de relacionamento entre clientes e fornecedores, o que implica em competências administrativas e jurídicas. Essas complexidades apresentam riscos para partes envolvidas e, como consequência, é comum a ocorrência de conflitos \cite{processoContratacao}.

%---------------------------------------------------------------------------------------------------------------------%

\section{Instrução Normativa Nº 04}

A Instrução Normativa nº 04 \citeonline{Normativa4} dispõe sobre o processo de contratação de Soluções de Tecnologia da Informação pelos órgãos integrantes do Sistema de Administração de Recursos de Tecnologia da Informação e Informática (SISP) do Poder Executivo Federal. Esta IN é a consolidação de um conjunto de boas práticas para Contratação de Solução de TI, que formam o Modelo de Contratações de Soluções de TI (MCTI).

A Instrução Normativa nº 04 \citeonline{Normativa4} está dividida em três capítulos:

\begin{easylist}[itemize]
& \textbf{Capítulo 1:} Diz respeito das disposições gerais.
& \textbf{Capítulo 2:} Diz respeito ao processo de contratação e é dividido em 3 seções.
					
					&& \textbf{Seção 1:} Diz respeito ao Planejamento da Contratação e é  								dividido em 4 Subseções.
					&& \textbf{Seção 2:} Diz respeito a Seleção do Fornecedor.
					&& \textbf{Seção 3:} Diz respeito a Gestão do Contrato e é  								dividido em 4 Subseções.

& \textbf{Capítulo 3:} Apresenta as Disposições Finais.
\end{easylist}


%---------------------------------------------------------------------------------------------------------------------%

%\section{Contrato do Órgão X}



%--------------------------------------------
