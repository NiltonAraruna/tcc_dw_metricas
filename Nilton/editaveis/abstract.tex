\begin{resumo}[Abstract]
 \begin{otherlanguage*}{english}

 		The internal quality of software is strictly connected with source code metrics. When these metrics are seen on spreadsheets which is the outcome from static code analysis, it does not show the sufficient visibility level for making technical decisions in a software project. At this work, a Data Warehousing environment was used in order to gather data related to source code metrics for supporting the decision making process. In spite of validation of Data Warehousing environment, it was applied on a case study which has evaluated the "Sistema Integrado de Gestão e Conhecimento", a information system from Instituto do Patrimônio Artístico Nacional (IPHAN), a Brazilian Department responsible for protecting cultural assets. The evaluation on case study has resulted in 12 quality intervals for source code metrics on 2 different configurations which were made from different softwares references. Additionally, it was found 317 code cleanup opportunities in 914 classes at last release of that software project.

   \vspace{\onelineskip}
 
   \noindent 
    \textbf{Palavras-chaves}: \textit{Source Code Metrics}. \textit{Data Warehousing}. \textit{Data Warehouse}
 \end{otherlanguage*}
\end{resumo}
